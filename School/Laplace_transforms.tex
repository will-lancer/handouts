\documentclass[11pt]{article}
\usepackage[margin=1in]{geometry}
\usepackage[pdftex]{graphicx}
\usepackage{amsmath,amssymb,amsthm}
\usepackage{william}
\usepackage{graphicx} %pictures
\usepackage{wrapfig} %for wrapping pictures
\usepackage{pgfplots} %graphs
\usepackage[makeroom]{cancel} %for striking out math
\usepackage{mathrsfs}
% \usepackage{draftwatermark} % for watermarks

% code for totaling up points
\usepackage{totcount}
\newtotcounter{tpts}
\newtotcounter{cpts}
\newtotcounter{endpts}
\newcommand{\clearpts}{\addtocounter{tpts}{\value{cpts}} \setcounter{cpts}{0}}
\newcommand{\pts}[1]{\clearpts \setcounter{cpts}{#1}}
\newcommand{\totpts}{\setcounter{endpts}{\totvalue{tpts} + \totvalue{cpts}}\arabic{endpts}}

% formatting for problems and solutions
\definecolor{ptred}{rgb}{0.7,0.1,0.1}
\newcommand{\ptfmt}[1]{\textbf{\color{ptred}#1\color{black}}}
\definecolor{advblue}{rgb}{0.1,0.1,0.7}
\newcommand{\advanced}{\!\textbf{\color{advblue}[A]\color{black}}\ }

\newtheorem*{solution}{Solution}
\newtheorem*{answer}{Answer}

% uncomment to hide solutions
% \usepackage{environ}
% \NewEnviron{hide}{}
% \let\solution\hide
% \let\endsolution\endhide
% \let\answer\hide
% \let\endanswer\endhide

% headers and footers
\usepackage{fancyhdr}
\pagestyle{fancy}
\lhead{Will Lancer}
\chead{}
\rhead{MAT 308 Notes: Laplace Transforms}
\lfoot{}
\cfoot{\thepage}
\rfoot{}
\renewcommand{\headrulewidth}{0.4pt}
\setlength{\headheight}{14pt}

\newcommand{\bigtitle}[1]{
    \begin{center}
    \huge \textbf{#1}
    \end{center}
}

\makeatletter
\newtheoremstyle{mystyle}
{\topsep}               % space above
{\topsep}               % space below
{}                      % bodyfont
{}                      % indent
{\bfseries}             % headfont
{}                      % punctuation
{0.6em}                 % space after head
{\llap{[\ptfmt{\arabic{cpts}}]\hspace{.6em}}\thmname{#1}\thmnumber{ #2}\thmnote{\normalfont{ (#3)}}{\bfseries .}}  %theoremheadspec
\theoremstyle{mystyle}
\newtheorem{pproblem}{Problem}
\makeatother

\linespread{1.03} % give a little extra room
\setlength{\parindent}{0.2in} % reduce paragraph indent a bit

%%%%%%%%%%%%%%%%%%%%%%%%%%%%%%%%%%%%%%%%%%%%%%
% \SetWatermarkText{\includegraphics{ghostfilip.png}}

\begin{document}

\bigtitle{MAT 308 Handouts: Laplace Transforms}

\noin
These are the first notes/handout that I will make for
MAT 308 (maybe the last too; we'll see how this goes).
They are on Laplace Transforms. Hopefully this helps
some people.
\textbf{Note:} if you see any errors (no
matter how small!), please let me know either via Discord
(user: $\verb|will.lancer|$), or by email ($\verb|william.lancer@stonybrook.edu|$).

\noin
\textbf{DISCLAIMER:} I have literally no idea what we
were taught in class, so this will most likely be off-topic
in some (a lot of?) areas. It should still be useful, though.

\section{The Basics}

\begin{psidea}{Laplace transforms}{}
    The \emph{Laplace transform} is a linear transform (also
    called an operator; same thing)
    that is defined to act on functions like so
    \begin{align*}
        \mathcal{L}[f(t)] = \tilde{f}(s) \equiv \int_0^\infty dt \, f(t) e^{-st}
    \end{align*}
    Note the change of \emph{argument} for the function $f(t)$
    when transforming to $\tilde{f}(s)$\footnote{If you're
    not used to seeing $\int dt \, f(t)$ as opposed
    to $\int f(t) \, dt$, just know the two are equivalent,
    and the former notation is the standard in theoretical
    physics.}.
\end{psidea}

\begin{psremark}{Some general comments about the Laplace transform}{}
    Some comments on Laplace transforms:
    \begin{itemize}
        \item The Laplace transform is a \emph{linear} transform/operator,
        which means it obeys homogeneity and addivity. Let $f, g$
        be functions, and $\alpha, \beta \in \reals$. 
        For a linear transform, we would have that,
        \begin{align*}
            \mathcal{L}[\alpha f(t) + \beta g(t)] & = \mathcal{L}[\alpha f(t)] + \mathcal{L}[\beta g(t)] \hspace{1.5cm} \text{(Additivity)}\\
            & = \alpha \mathcal{L}[f(t)] + \beta \mathcal{L}[g(t)] \hspace{1.5cm} \text{(Homogeneity)}\\
            & = \alpha \tilde{f}(s) + \beta \tilde{g}(s).
        \end{align*}
        \item Again, note the change in argument for the function;
        it goes from $t \to s$. This is non-trivial, as the function
        is now pulling its values from a different \emph{domain}
        than before. This is incredibly useful in physics,
        where Laplace transforms are often used to switch
        between the time and frequency domains.
        \item Note that the Laplace transform is invertible
        for ``nice''\footnote{I'll actually define what this means later.} functions.
        We know that an invertible map is just another word for 
        an \emph{isomorphism}; this tells us that, even though
        the functions may seem quite different on the surface
        of it, they actually still maintain the same \emph{structure}
        as one another.
    \end{itemize}
\end{psremark}

\begin{psmotivation}{Why should we care about Laplace transforms?}{}
    A reasonable thing to ask is, ``Why do we even care about
    Laplace transforms''? The reason is that, without overstating,
    \begin{quote}
        \emph{Laplace transforms are the most powerful general technique
        for solving ordinary differential equations there is.}
    \end{quote}
    If you had the choice for only \emph{one} method for solving
    differential equations, you would be best off if you chose
    Laplace transforms. Due to their structure, and their
    easy rules, they can easily deal with extremely complicated
    ordinary differential equations.
    The main strength of Lapace transforms is
    that \emph{they turn differential equations
    into algebra}. As you know, algebra and basic
    integration\footnote{Better yet, just looking
    stuff up in a table} is much easier to do than solving
    differential equations. Another bonus for Laplace
    transforms is, as you will see later, the boundary
    conditions for the differential equation are \emph{built
    into} the transform. Common Laplace transforms W.
\end{psmotivation}

\noin
\color{black} I think the best way to get a feel for this
is through examples. Here are the canonical ones:

\begin{psexample}{}{}
    \begin{hwproblem}
        Find the Laplace transform for $1$. Generalize this
        to any constant.
    \end{hwproblem}
    \begin{solution}
        We use the definition of the Laplace transform.
        Recall
        \begin{align*}
            \mathcal{L}[f(t)] = \tilde{f}(s) \equiv \int_0^\infty dt \, f(t) e^{-st}.
        \end{align*}
        For $f(t) = 1$, we have, (remember that you have to take the limit
        of upper bound to infinity instead of actually ``integrating
        at infinity'')\footnote{I will be sloppy with doing this 
        for the rest of the handout.}
        \begin{align*}
            \tilde{f}(s) & = \lim_{B \to \infty} \int_0^B dt \, e^{-st}\\
            & = \lim_{B \to \infty} - \left[ \frac{e^{-st}}{s} \right]_0^B.
        \end{align*}
        If $s > 0$, then this integral converges. Thus,
        we only define our integral for $s > 0$.
        We would then have that,
        \begin{align*}
            \tilde{f}(s) & = \lim_{B \to \infty} - \left[ \frac{e^{-st}}{s}\right]_0^B\\
            & = \lim_{B \to \infty} \left[ \frac{e^{-st}}{s} \right]_B^0\\
            & = \frac{1}{s} - \frac{0}{s} = \frac{1}{s}.
        \end{align*}
        Thus our answer for the Laplace transform of $f(t) = 1$
        is $\mathcal{L}(f(t)) = \tilde{f}(s) = 1/s$.
        To generalize this to any constant, use the
        \emph{linearity} of the Laplace transform to your
        advantage. Let $a \in \reals$.
        \begin{align*}
            \mathcal{L}[a] & = \tilde{f}(s) = \int_0^\infty dt \, a e^{-st}\\
            & =  a \int_0^\infty dt \, e^{-st} = a\mathcal{L}(1) = \frac{a}{s}.
        \end{align*}
        Which is our answer. Notice how powerful linearity is; we will continually
        use this property to our advantage in the future.
    \end{solution}
    \begin{hwproblem}
        What is the Laplace transform of $e^{at}$ where $a \in \complex$?
    \end{hwproblem}
    \begin{solution}
        This is going to be one of the most important Laplace
        transform identities we will ever use.
        We're actually going to answer
        a \emph{different} question instead. What
        is the value of
        \begin{align*}
            \mathcal{L}[e^{at}f(t)],
        \end{align*}
        for some arbitrary function $f(t)$? Let's
        give it a shot. Using the definition of
        the Laplace transform gives us,
        \begin{align*}
            \mathcal{L}[e^{at}f(t)] = \tilde{f}(s) 
            & = \int_0^\infty dt \, e^{at}f(t) e^{-st}\\
            & =  \int_0^\infty dt \, f(t) e^{-(s - a)t}\\
            & = \tilde{f}(s - a).
        \end{align*}
        Which is our answer. Notice the change of variables
        this gives us; this is due to the additivity of exponents
        for multiplying exponentials. Notice that if we let $f(t) = 1$,
        we would be able to find the Laplace transform for $e^{at}$.
        Recall that $\mathcal{L}[1] = \tilde{f}(s) = 1/s$.
        Computing $\mathcal{L}[e^{at}f(t)]$ with $f(t) = 1$
        gives us,
        \begin{align*}
            \mathcal{L}[e^{at}f(t)] = \mathcal{L}[e^{at}(1)]
            = \tilde{f}(s - a) = \frac{1}{s - a}, \hspace{0.5cm} s > 0
        \end{align*}
        Which is our answer. As you can see, this general
        method is quite powerful for computing Laplace transforms
        of general exponentials. This process is sometimes
        called \emph{s-shift}.
    \end{solution}
    \begin{hwproblem}
        What are the Laplace transforms of $\cos{(t)}$ and $\sin{(t)}$?
    \end{hwproblem}
    \begin{solution}
        Hmmmm. This feels difficult. Trying to integrate something like,
        \begin{align*}
            \mathcal{L}[\cos{(t)}] & = \tilde{f}(s) = \int_0^\infty dt \, \cos{(t)}e^{-st}
        \end{align*}
        seems kind of ugly. Question: is there any way
        we can \emph{rewrite} $\cos{(t)}$ to somehow
        make use of our previous techniques?
        Preferably something to do with exponentials
        and constants, because they're easy.
        The answer is \emph{yes}, and it comes
        from a familiar identity:
        \begin{align*}
            \textbf{Euler's Identity:} \hspace{1cm} e^{i t} = \cos{(t)} + i \sin{(t)}.
        \end{align*}
        Solving for cosine from this gives us,
        \begin{align*}
            \cos{(at)} & = \frac{e^{iat} - e^{-iat}}{2}.
        \end{align*}
        Well well well$\ldots$ we know how to solve
        this! We know the Laplace transform of $e^{iat}$
        is (by the previous question),
        \begin{align*}
            \mathcal{L}[e^{iat}] & = \tilde{f}(s - ia) = \frac{1}{s - ia}.
        \end{align*}
        We now have everything we need to solve our
        problem.
        \begin{align*}
            \mathcal{L}[\cos{(at)}] & = \mathcal{L}\left[ \frac{e^{iat} - e^{-iat}}{2} \right]\\
            & = \frac{1}{2} \left( \mathcal{L}[e^{iat}] + \mathcal{L}[e^{-iat}] \right) \hspace{1cm} \text{(Linearity)}\\
            & = \frac{1}{2} \left( \frac{1}{s - ia} + \frac{1}{s + ia} \right)\\
            & = \frac{1}{2} \left( \frac{2s}{s^2 + a^2}\right)\\
            & = \frac{s}{s^2 + a^2}, \hspace{0.5cm} \Re{s} > 0
        \end{align*}
        Which is our answer. We can follow an almost identical
        process for sine, which the reader is encouraged
        to do (the following relation might help $\winky$)
        \begin{align*}
            \sin{(at)} = \frac{e^{iat} - e^{-iat}}{2i}.
        \end{align*}
        Going through the calculations gives us (as
        the reader should verify),
        \begin{align*}
            \mathcal{L}[\sin{(t)}] = \tilde{f}(s - a) = \frac{a}{s^2 + a^2}, \hspace{0.5cm} \Re{s} > 0
        \end{align*}
        We can also derive the same results by taking
        the real and imaginary parts of $e^{ia}$,
        which is an exercise for the reader;
        I will put them in formal problems so
        you're more motivated to do it.
    \end{solution}
        \begin{hwproblem}
            Derive the Laplace transform for $\cos{(t)}$ by taking
            the real part of the Laplace transform for $e^{ia}$.
        \end{hwproblem}
        \begin{hwproblem}
            Derive the Laplace transform for $\sin{(t)}$ by taking
            the imaginary part of the Laplace transform for $e^{ia}$.
        \end{hwproblem}
        I'm watching you$\ldots$ you better do these problems $\straightface$.
\end{psexample}

\section{Solving differential equations} 
Using Laplace transforms to solve differential
equations is a three-step process,
\begin{align*}
    \text{Transform} \to \text{Rearrange} \to \text{Inverse Transform}.
\end{align*}
The only part you don't know yet is the inverse
transform, which we'll learn now. Btw, the ``rearrange''
part is just algebra, as you'll see.
\begin{psidea}{Inverse Laplace Transform}{}
    The inverse Laplace transform is just the process
    of going from $\tilde{f}(s) \to f(t)$. You do
    this by recognizing what the Laplace transform
    is, and then making the proper inverse transform.
\end{psidea}

\begin{psidea}{Partial fractions}{}
    The main way you're going to get the Laplace
    transform into a form that you can easily
    recognize is by simplifying rational functions,
    which you should remember from middle/high school.
    If you don't remember how to do this, read \href{https://www.mathsisfun.com/algebra/partial-fractions.html}{this}.
\end{psidea}

\begin{psremark}{When do Laplace transforms exist?}{}
    When do Laplace transforms exist? This is a reasonable
    question, because \emph{in the definition} of the Laplace
    transform is an integral to infinity; these kinds of integrals
    sometimes have the bad habit of equating to infinity.
    So, when are they defined? The answer is when the
    function you're transforming, $f(t)$, is of
    \emph{exponential order}. What does this mean?
    It means that the scaling of $f(t)$ with $t$
    can be expressed as,
    \begin{align*}
        |f(t)| \leq Ce^{kt}.
    \end{align*}
    This is equivalent to saying that
    \begin{align*}
        I = \int_0^\infty dt \, |f(t) e^{-st}| < \infty.
    \end{align*}
    Waterman probably won't give us examples that don't
    converge, but it's important to know that you can't
    apply the Laplace transform to everything (just
    most things).
\end{psremark}

\begin{psexample}{Some easy examples}{}
    \begin{hwproblem}
        Find the inverse Laplace transform of
        \begin{align*}
            \frac{12}{s}.
        \end{align*}
    \end{hwproblem}
    \begin{solution}
        Recognizing this as the Laplace transform of
        $12$ gives us our answer,
        \begin{align*}
            \frac{12}{s} = 12 \cdot \frac{1}{s} = \mathcal{L}[12] \implies \mathcal{L}^{-1}[12/s] = 12.
        \end{align*}
    \end{solution}
    \begin{hwproblem}
        Find the inverse Laplace transform of
        \begin{align*}
            \frac{69}{s^2 + s - 6}
        \end{align*}
    \end{hwproblem}
    \begin{solution}
        We must first do partial fraction decomposition.
        Doing so gives us,
        \begin{align*}
            \frac{69}{s^2 + s - 6} = \frac{69}{(s + 3)(s - 2)} \implies 69 = A(s - 2) + B(s + 3)
        \end{align*}
        Solving for this gives us $A = -69/5$ and $B = 69/5$.
        Our problem then turns into,
        \begin{align*}
            \frac{69}{s^2 + s - 6} = \frac{-65}{5(s + 3)} + \frac{69}{5(s - 2)}.
        \end{align*}
        Recognizing these as exponetials gives us our answer,
        \begin{align*}
            \frac{-65}{5(s + 3)} + \frac{69}{5(s - 2)} = - \frac{69}{5} \frac{1}{s + 3} + \frac{69}{5} \frac{1}{s - 2} = - \frac{69}{5} \mathcal{L}[e^{-3t}] + \frac{69}{5} \mathcal{L}[e^{2t}]\\
            \implies
            \mathcal{L}^{-1}\left[ \frac{-65}{5(s + 3)} + \frac{69}{5(s - 2)} \right] = - \frac{69}{5} e^{-3t} + \frac{69}{5} e^{2t}.
        \end{align*}
    \end{solution}
\end{psexample}

\begin{psidea}{Laplace transform of a derivative}{}
    The Laplace transform of the time derivative
    of a function $f(t)$ is defined as,
    \begin{align*}
        \mathcal{L}[f'(t)] = s\tilde{f}(s) - f(0).
    \end{align*}
    This can be recursively applied to give a formula
    for $n$ time derivatives,
    \begin{align*}
        \mathcal{L}[f^{(n)}(t)] = s^n \tilde{f}(s) + \sum_{k = 1}^{n} (-1) s^{n - k} f^{k - 1}(0).
    \end{align*}
\end{psidea}

\begin{psexample}{Time derivatives}{}
    \begin{hwproblem}
        Prove the two formulas given in the idea above. \textbf{Hint:} use
        integration by parts.
    \end{hwproblem}
    \begin{solution}
        I'm going to show the first one; the second one follows
        by induction applied to the approach you take to solve
        for the first. Integrating by parts gives us,
        \begin{align*}
            \mathcal{L}[f'(t)] & = \int_0^\infty dt \, f'(t) e^{-st}\\
            & = [e^{-st} f(t) ]_0^\infty - (-s) \int_0^\infty dt \, f(t) e^{-st}\\
            & = s\tilde{f}(s) - f(0).
        \end{align*}
        The reader should prove the second part. Feel free to
        not even do induction; just compute the case of $\mathcal{L}[f''(t)]$
        and guess a generalization from there.
    \end{solution}
\end{psexample}


\begin{psexample}{A less trivial example}{}
    \begin{hwproblem}
        Find the specific solution to,
        \begin{align*}
            x''(t) + x(t) = \cos{2t}
        \end{align*}
        with $x(0) = 0$ and $x'(0) = 1$.
    \end{hwproblem}   
\end{psexample}

\begin{pssolution}{}{}
    Laplace transform both sides,
    \begin{align*}
       \mathcal{L}[x''(t) + x(t)] = \mathcal{L}[\cos{2t}]\\
       s^2 \tilde{x}(s) - sx(0) + x'(0) + \tilde{x}(s) = \frac{s}{s^2 + 4}.
    \end{align*}
    Plug in the initial conditions,
    \begin{align*}
        s^2 \tilde{x}(s) - 1 + \tilde{x}(s) = \frac{s}{s^2 + 4}.
    \end{align*}
    Solve for $\tilde{x}(s)$ and use partial fractions,
    \begin{align*} 
        \tilde{x}(s) & = \frac{s}{(s^2 + 1)(s^2 + 4)} + \frac{1}{s^2 + 1}\\
        & = \frac{1}{3} \frac{s}{s^2 + 1} - \frac{1}{3}\frac{s}{s^2 + 4} + \frac{1}{s^2 + 1}.
    \end{align*}
    Inverse transforming this gives us,
    \begin{align*}
        x(t) = \frac{1}{3}\cos{t} - \frac{1}{3} \cos{(2t)} + \sin{t}.
    \end{align*}
\end{pssolution}

\section{Convolution}

I'm going to throw the definition of
convolution at you, and then explain it.
\begin{psidea}{Convolution}{}
    We define the convolution of two functions $f(t)$
    and $g(t)$ as follows,
    \begin{align*}
        f * g \equiv \int_0^t du \, f(u) g(t - u).
    \end{align*}
    Although it's not evident in the definition,
    convolution is commutative, so $f * g = g * f$.
\end{psidea}

\begin{psremark}{Some convoluted comments about convolution}{}
    \begin{itemize}
        \item Where the hell did that definition come from? To answer
        that question, let's answer a different question:
        \begin{hwproblem}
            What function $h(t)$ would I have to Laplace 
            transform to have $\tilde{h}(s) = \tilde{f}(s)\tilde{g}(s)$?
            As in, what $h(t)$ gives us the following equivalence?
            \begin{align*}
                \int_0^\infty dt \, h(t) e^{-st} = \tilde{f}(s)\tilde{g}(s).
            \end{align*}
        \end{hwproblem}
        \begin{solution}
        Hmmm. Let's go back to the definitions of $\tilde{f}(s)$ and $\tilde{g}(s)$
        and see if anything pans out.
        \begin{align*}
            & \tilde{f}(s) \equiv \int_0^\infty du \, f(u)e^{-su} & \tilde{g}(s) \equiv \int_0^\infty dv \, g(v)e^{-sv}. 
        \end{align*}
        We let the integration variable not be $t$ for a reason you will
        see in a second; it doesn't matter what it is, as it's a dummy variable.
        Take the product of these functions,
        \begin{align*}
            \tilde{f}(s)\tilde{g}(s) & = \int_0^\infty du \, f(u)e^{-su} \cdot \int_0^\infty dv \, g(v)e^{-sv}\\
            & = \int_0^\infty dv \int_0^\infty du \, f(u)g(v)e^{-s(u + v)}.
        \end{align*}
        We want to make this double integral look something like a Laplace transform,
        so let's make a change of variables, $t = u + v$. Subbing this in gives us,
        \begin{align*}
            \int_0^\infty dv \int_0^\infty du \, f(u)g(v)e^{-s(u + v)}
            & = \int_0^\infty dt \underbrace{\int_0^t du \, f(u)g(t - u)}_{h(t)} e^{-st}.
        \end{align*}
        As you can see, if we define $h(t) \equiv \int_0^t du \, f(u) g(t - u)$,
        we get our desired result: $\mathcal{L}[h(t)] = \tilde{f}(s)\tilde{g}(s)$.
        Very cool.
        \end{solution}
        \item The commutativity of the convolution is clear
        from the secondary meaning of the convolution; $f * g$
        is a function such that $\mathcal{L}[f * g] = \tilde{f}(s)\tilde{g}(s)$.
        As you can see, if we switch $f * g$ to $g * f$, 
        we simply commute the terms on the RHS, and this 
        is equivalent to the original convolution.
        \begin{align*}
            \mathcal{L}[f * g] = \tilde{f}(s)\tilde{g}(s) = \tilde{g}(s)\tilde{f}(s) = \mathcal{L}[g * f].
        \end{align*}
    \end{itemize}
\end{psremark}

\begin{psexample}{Some examples}{}
    \begin{hwproblem}
        Find the convolution of $t^2$ and $t$.
    \end{hwproblem}
    \begin{solution}
        Using the definition gives us,
        \begin{align*}
            f * g = t^2 * t & = \int_0^t du \, u^2 (t - u)\\
            & = \int_0^t du \, u^2t - u^3\\
            & = \frac{u^3 t}{3} - \frac{u^4}{4} \Bigg|_0^t = \frac{t^4}{12}.
        \end{align*}
    \end{solution}
    \begin{hwproblem}
        Find the convolution of $f(t)$ and $1$.
    \end{hwproblem}
    \begin{solution}
        Using the definition gives us,
        \begin{align*}
            f * g = f(t) * 1 & = \int_0^t du \, f(u)\\
            & = F(t) - F(0).
        \end{align*}
    \end{solution}
\end{psexample}

\section{Parting comments}

Hopefully you're more comfortable with Laplace
transforms now as compared to when you started
this handout. I want to emphasize that Laplace
transforms are \emph{simple}; they have fairly
few computational rules, and the annoying stuff
can just be looked up in a table somewhere.
They're one of the most powerful tools you're
going to have in your differential equations
toolbelt from now on; make use of it.
Good luck on the MAT 308 homework \salute.


\end{document}
